\documentclass[11pt]{article}

% Packages
\usepackage[utf8]{inputenc}
\usepackage{amsmath, amssymb}
\usepackage{graphicx}
\usepackage{hyperref}
\usepackage{geometry}
\usepackage{natbib}
\usepackage{setspace}
\usepackage{titlesec}

% Geometry settings
\geometry{
	a4paper,
	total={6in, 9in},
	left=1.25in,
	top=1in,
}

% Title settings
\titleformat{\section}{\normalfont\Large\bfseries}{\thesection}{1em}{}
\titleformat{\subsection}{\normalfont\large\bfseries}{\thesubsection}{1em}{}

% Title
\title{An Experimental Analysis of the Influence of the Precision of
	Parameters in an Artificial Neural Network}
\author{Denise Katritschenko, Elias Reich, Sayed Abozar Sadat, Lukas Schwaiger\\
	\small University of Salzburg\\	
	\small Department of Artificial Intelligence and Human Interfaces (AIHI) }

\date{\today}

\begin{document}
	
	\maketitle
	
	\begin{abstract}
		TODO
	\end{abstract}
	
	\section{Introduction}
	Artificial Neural Networks (ANNs) are a key part of today’s machine learning systems. They’ve made huge progress in areas like image recognition, speech processing, and self-driving cars. However, one of the biggest challenges with ANNs is that they use a lot of memory and computing power. This becomes a problem when you want to run them on devices with limited resources, such as smartphones, embedded systems, or edge devices. \\ \\
	To solve this, researchers have looked into ways to make neural networks more efficient. A common approach is to reduce the size of the network or shrink the input batch size. But doing this can harm the network’s ability to learn and generalize. A better strategy is to reduce the numerical precision of the network’s parameters such as things like weights, activations, and gradients. This method, called quantization, makes the model smaller and faster while using less energy, without necessarily hurting accuracy. \\ \\
	One advanced version of this method is called mixed-precision quantization. Instead of using the same number of bits for every layer in the network, mixed-precision assigns different bitwidths depending on how important each part is. For example, the early layers that handle low-level features might use higher precision, while deeper layers could work fine with fewer bits. This flexible approach often leads to better results than using a single fixed precision for the whole model. Researchers have used a range of techniques, like gradient descent, heuristics, and even evolutionary algorithms, to find the best bitwidth settings for each layer. \\ \\
	Another method that helps reduce resource usage is network pruning. This means cutting out parts of the network that don’t contribute much to its performance. One interesting way to do this is with simulated annealing, a search algorithm that mimics the process of cooling metal to find a stable structure. It doesn’t need gradient information and can still find good network shapes by exploring many possibilities. This makes it useful when you want to shrink a network without going through heavy retraining. \\ \\
	Previous studies have shown that reducing the precision of neural network parameters can lead to significant improvements in memory usage and processing speed. Even when precision is reduced to very low levels, models can still perform competitively on complex tasks. However, while these early results are promising, the overall effects of lowering precision, especially how it influences model accuracy, training behavior, and storage efficiency, are still not fully understood and deserve more detailed investigation. \\ \\
	This is where our experimental work begins. 
	In this paper, we take a closer look at how different levels of parameter precision affect the behavior and performance of neural networks. Our goal is to understand what happens when we reduce precision in various ways, and how much we can lower it before performance starts to drop. \\ \\
	We start by testing how reducing precision during both training and inference affects model accuracy and runtime. Next, we look at what happens when we train the network using full precision, but lower the precision only during inference. This setup is common in real-world applications, where a powerful machine handles training, but the final model is deployed on a lightweight device. 
	Then, we run a series of tests where we round the trained weights to a limited number of decimal places. This helps us see how much we can simplify the stored weight values without hurting accuracy too much. This is especially useful for devices with very limited storage. 
	We also try out mixed-precision training, using a mix of FP16 and FP32 formats. On top of that, we experiment with different types of gradient clipping, like global norm clipping and value clipping, to see how they affect training stability and final accuracy. 
	Even though we use relatively simple network architectures and datasets, our results show that neural networks are quite robust to low-precision settings. In some cases, we were able to round weights to just three decimal digits and still get nearly the same accuracy as the full-precision version. 
	These findings show that it’s possible to significantly reduce the memory and compute needs of neural networks without losing much performance. This makes them more practical for use in low-resource settings. It also opens the door for more advanced techniques that combine quantization, pruning, and precision-aware training to build highly efficient models for real-world applications. 
	
	
	
	
	\bibliographystyle{plainnat}
	\bibliography{references}
	
\end{document}
